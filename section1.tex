Kisälliopetus on uusi ja innovatiivinen tapa opettaa matematiikkaa.
Se eroaa merkittävästi perinteisestä matematiikan opetuksesta, jossa uusi asia luennoidaan ja kotiläksyksi annetaan tehtäviä, jotka liittyvät tunnilla käytyihin asioihin.
Kisälliopetuksessa korostetaan laskemisen merkitystä.
Tehtäviä on tyypillisesti enemmän, ja opettaja kiertelee ja auttaa oppilaita tekemään laskuja.
Neuvonnalla pyritään siihen, että oppilas ymmärtää ongelman ja mahdolliset ratkaisumenetelmät -- ei siis niin, että menetelmät tai ratkaisut vain kerrottaisiin opiskelijalle.
Näin opiskelija itse päätyy ratkaisuun, ja samalla oppii "oikeasti" ymmärtämään käytetyn menetelmän.
Kisälli\-opetuksessa opiskelija pääsee osaksi asiantuntijakulttuurin tapaa toimia.\cite{hautala2012extreme,vihavainen2011extreme}

Helsingin Normaalilyseossa on yksi matematiikan kurssi, joka täyttää kisälli\-opetuksen piirteet.
Kurssi MAA15S, harjoituskurssi, on suunniteltu käytäväksi samanaikaisesti muiden pitkän matematiikan kurssien kanssa.
Niinpä harjoituskurssilla käsitellään jo opittuja asioita syventävästi.
Kurssin järjestämistapa tarjoaa myös hyvät mahdollisuudet eriyttämiseen.

Harjoituskurssi tapaa kerran viikossa puolen vuoden ajan, ja opiskelija saa siitä yhden kurssisuorituksen.
Kurssilla ei tästä syystä voida tehdä varsinaisen matematiikan kurssin laskuharjoitustehtäviä, vaan kurssin järjestävä opettaja valmistelee laskettavaksi tehtävämonisteen.
Ryhmätyö on sallittua, eikä kurssilla ole tehtäväkiintiötä.
Niinpä kurssin ilmapiiri on hyvä ja rento.
Toisaalta oppilailla saattaa esiintyä motivaatio-ongelmia.
Tutkimuksen tekemisen aikana emme huomanneet tämän muodostuvan suureksi ongelmaksi.
