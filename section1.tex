Kisälliopetus on uusi ja innovatiivinen tapa opettaa matematiikkaa. Aihe on ajankohtainen, sillä sen tiimoilta on tehty lähivuosina tutkimusta Helsingin yliopistossa sekä matematiikan- ja tilastotieteen laitoksella että tietojenkäsittelytieteen laitoksella.
Se eroaa merkittävästi perinteisestä matematiikan opetuksesta, jossa uusi asia luennoidaan ja kotiläksyksi annetaan tehtäviä, jotka liittyvät tunnilla käytyihin asioihin.
Kisälliopetuksessa korostetaan laskemisen ja tekemällä oppimisen merkitystä. Opiskelijan ajatellaan sisäistävän opittava asia paremmin ja syvällisemmin, jos hän saa itse tehdä sen sijaan, että joku toinen kertoisi sen hänelle.

Tehtäviä on tyypillisesti enemmän, ja opettaja kiertelee ja auttaa oppilaita tekemään laskuja. Auttamisessa pääpaino on siinä, että opiskelija johdatellaan itse ymmärtämään asia, esimerkiksi esittämällä hänelle sopivia kysymyksiä. Näin opiskelija itse päätyy ratkaisuun, ja samalla oppii "oikeasti" ymmärtämään käytetyn menetelmän.
Neuvonnalla pyritään siihen, että oppilas ymmärtää ongelman ja mahdolliset ratkaisumenetelmät -- ei siis niin, että menetelmät tai ratkaisut vain kerrottaisiin opiskelijalle.
Kisälli\-opetuksessa opiskelija pääsee osaksi asiantuntijakulttuurin tapaa toimia.\cite{hautala2012extreme,vihavainen2011extreme}
Tärkeä osa opiskelua on myös opiskelijoiden keskinäinen vuorovaikutus. 

Helsingin Normaalilyseossa on yksi matematiikan kurssi, joka täyttää kisälli\-opetuksen piirteet.
Kurssi MAA15S, harjoituskurssi, on suunniteltu käytäväksi samanaikaisesti muiden pitkän matematiikan kurssien kanssa. 
Harjoituskurssi tapaa kerran viikossa puolen vuoden ajan, ja opiskelija saa siitä yhden kurssisuorituksen.
Kurssilla ei tästä syystä voida tehdä varsinaisen matematiikan kurssin laskuharjoitustehtäviä, vaan kurssin järjestävä opettaja valmistelee laskettavaksi tehtävämonisteen kutakin kertaa varten, ja opiskelijat laskevat näitä tehtäviä kurssilla kahden tunnin ajan.
Kurssin järjestämistapa tarjoaa myös hyvät mahdollisuudet eriyttämiseen, sillä jokaisella on mahdollisuus saada yksilöllistä ohjausta ja valita tehtävät oman taitotasonsa mukaan.

Harjoituskurssilla myös syvennetään jo opittuja asioita, sekä harjoitellaan laskimen käyttöä.
Ryhmätyö on sallittua, jopa suotavaa, eikä kurssilla ole tehtäväkiintiötä. 
Ilmapiiri kurssilla on rento, tehtävien tekoa tai opiskelijoiden käytöstä ei kontrolloida, vaan he saavat omaan tahtiinsa laskea ja keskustella tehtävistä.
Toisaalta asiassa on myös kääntöpuoli, jollain opiskelijoilla oli omien sanojensa mukaan vaikeuksia motivoitua tehtävien tekoon. Kuitenkaan tutkimuksen tekemisen aikana emme huomanneet tämän muodostuvan suureksi ongelmaksi.
