Tutkimuksemme tutkimustehtävänä oli selvittää, kuinka pajatyöskentely sopii lukio-opintojen yhteyteen. Perinteisessä opetuksessa ei yleensä korosteta oma-aloitteisuutta ja opiskelijan omaa vastuuta oppimiseen ja tiedon hankkimiseen, vaan opiskelijat on totutettu saamaan vastaukset valmiina. Halusimme siis tutkia, toimiiko yliopisto-opinnoista tuttu pajamuotoinen opiskelu myös lukioympäristössä.
Halusimme saada oppilaiden mielipiteet ja näkemykset kurssista esiin. Päämääränämme oli selvittää, miten opiskelijat suhtautuvat kurssiin, millaisia tavoitteita heillä oli kurssilla osallistumisen suhteen.
Niinpä emme halunneet tutkia harjoituskurssin vaikutusta muiden kurssien arviointiin, vaan haastatella opiskelijoita suoraan. Kysyimme oppilaiden mielipiteitä harjoituskurssin hyödyllisyydestä.
Tutkimuskysymykseksemme muodostui siis seuraava:
\textbf{Kuinka hyödylliseksi lukion oppilaat kokevat laskupajatyöskentelyn?}

