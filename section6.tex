Tutkimamme harjoituskurssin kaltaisia kursseja ei juuri järjestetä muissa lukioissa.
Matemaattinen harjaantuminen jää monesti itsenäisesti tehtävien kotitehtävien varmaan.
Niinpä näemme, että nykylukiossa on paikkansa tutkimamme kaltaiselle harjoituskurssille.%samankaltaiselle harjoituskurssille on paikkansa.
On kuitenkin varottava, että matematiikan oppiminen ei redusoidu pelkäksi laskemiseksi.

\subsection{Kehittämisehdotuksia}
Harjoituskurssin tehtävät edustivat mielestämme pitkän matematiikan perustehtäviä.
Varsinaisia ongelmatehtäviä ei juuri ollut, ja sanallisten tehtävien tilanteet olivat usein ''tekemällä tehtyjä``.
Harjoituskurssia voisi parantaa lisäämällä tehtävien joukkoon avoimia tutkimustehtäviä, pohdintaa ja jopa ymmärrystä syventäviä käsitteitä.
Helsingin normaalilyseon oppilailla on käytössään tietokoneet.
Tämä avaa ovet interaktiivisen materiaalin käyttöön, Geogebratutkimuksiin ja tietokoneavusteiseen matematiikkaan.

Osa opiskelijoista koki kurssin tylsänä.
Tälle opiskelija-ainekselle tyypillistä oli se, että he kokivat osaavansa matematiikkaa jo valmiiksi hyvin.
Tämän osan opiskelijoista saamiseksi mukaan voi eriyttää tehtäviä tai ottaa mukaan myös edellä mainittuja avoimia tehtäviä.

\subsection{Ongelmat tutkimuksessa}
Osallistujia harjoituskurssilla oli 19, joten aineiston luotettavuus saattaa olla kyseenalainen.
Toisaalta avokysymyksillä ja opetukseen osallistumalla uskomme, että saimme hyvän kuvan kurssin luonteesta.
Tämän lisäksi oppilasaines on valikoitunutta Helsingin normaalilyseossa, joten uskomme, että tuloksiamme ei voida yleistää kaikkiin lukioihin.

Tutkimuskysymystä ja kyselylomaketta tarkemmin tarkastellen olisi myös voinut lisätä tutkimuksen tarkkuutta.

