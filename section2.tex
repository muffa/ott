\subsection{Lähikehityksen vyöhyke}
Psykologi Les Vygotskin teoria lähikehityksen vyöhykkeestä esittää, että henkilön toimiessa häntä itseä kokeneemman ohjaajan vaikutuspiirissä, hän kykenee suorituksiin, jotka ylittävät hänen nykyisen eli aktuaalisen tieto- ja taitotasonsa. Vygotskin näkemyksen mukaan henkilön aktuaalisen taitotason ja potentiaalisen taitotason väliin sijoittuu lähikehityksen vyöhykkeeksi nimitetty tila, missä henkilö kykenee ympäristön vaikutuksen ansiosta hänen potentiaalisen taitotasonsa mukaisiin saavutuksiin.\cite{vygotsky1978mind}
\subsection{Scaffolding}
Scaffolding on syväoppimiseen tähtäävä lähikehityksen vyöhykkeen pedagoginen sovellus. Siinä opettaja haastaa opiskelijan tekemään hieman hänen aktuaalista taitotasoaan vaativampia tehtäviä, jotka opiskelija tekee opettajan ohjauksessa. Oppiminen on tehokkaampaa ja syvempää, kuin jos tehtävä olisi liian haasteellinen tai selvästi hänen taitotasoaan alempana. Ideana on, että ohjaaja johdattelee opiskelijan ratkaisemaan tehtävän itse, ja välttää kaikin tavoin ratkaisemasta sitä hänen puolestaan. Yksi scaffoldingin tavoitteita on, että opiskelija vähitellen kehittäisi itselleen opiskelu- ja ongelmanratkaisustrategioita. Keskeistä on myös, että opiskelija oppii paitsi vastaamaan, myös esittämään kysymyksiä.\cite{kirschnerswellerclark}
\subsection{Ongelmaperustainen oppiminen}
Näistä löytyy selvä kytkös ongelmaperustaiseen oppimiseen. Ongelmaperustainen oppiminen on pedagoginen suuntaus, joka tarjoaa kiinnostavan lähesty\-mistavan myös matematiikan opetukseen. Ongelmaperustaisessa oppimisessa opiskelija ratkoo ongelmia, missä hänen on sovellettava laajasti aikaisemmin hankittua teoriapohjaa. Hänelle ei välttämättä ole täysin selvää, mitä eri tekniikoita ratkaisemisessa on sovellettava. Toinen ongelmaperustaisen oppimisen lähestymistapa on tarjoilla teoria ongelman muodossa: opis\-kelija saa eteensä ongelman, ja hänen on löydettävä sen ratkaisemiseen vaadittava teoria ja heuristiikat.\cite{schoenfeld}
\subsection{Kisälliopetus}
Laskupajaopetus voidaa nähdä sovelluksena niin sanotusta kisälliopetuk\-sesta. Kisälliopetus on ajankohtainen tutkimusaihe, jota on tutkittu muun muassa Helsingin yliopistossa yliopistotason kursseilla. Matematiikan opiske\-lussa tehtävien tekeminen on ratkaisevassa asemassa ja sitä ei voi oppia pelkästään lukemalla tai kuuntelemalla. Lukio-opetuksessakin painotus on vahvasti tehtävien tekemisessä. Pyritään siihen, että opiskelijat ratkoisivat mahdollisimman paljon tehtäviä. Tehtävien pedagoginen hyöty ja potentiaali jää kuitenkin käyttämättä, mikäli ne ovat niin haasteellisia opiskelijan taitoihin nähden, että niiden tekeminen ei yksinkertaisesti onnistu. Mate\-ma\-tiikka on kuin lohikäärme, jonka kanssa tulee taistella, mutta taitavinkin soturi on joskus tarvinnut mestarin.
\subsection{Matematiikkapelko}
\textbf{...tähän}

