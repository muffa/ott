Aineiston pienuuden vuoksi päätimme käyttää laadullista tutkimusstrategiaa aineistoa kootessa.

Jaoimme laskupajassa läsnäolleille opiskelijoille ($n=19$) kyselylomakkeet, ja opiskelijat arvioivat seitsemää väitettä sekä Likert-asteikolla että sanallisesti.
Lomakkeessa oli myös viisi avointa kysymystä.

Oppilaat arvioivat seuraavia väittämiä:
\begin{itemize}
\item Harjoituskurssi on parantanut menestystäni muilla matematiikan kursseilla.
\item Tunnen nykyään osaavani ja ymmärtäväni matematiikkaa paremmin.
\item Harjoituskurssi on lisännyt itsevarmuuttani matematiikan osaamisen suhteen.
\item Harjoituskurssista on ollut minulle hyötyä.
\item Saan harjoituskurssilla apua tehtävien ratkaisemiseen.
\item Harjoituskurssilla on mukavaa.
\item Saan harjoituskurssilla tehtyä sellaisiakin tehtäviä, joita en itenäisesti osaisi tehdä.
\end{itemize}
Lomakkeessa oli seuraavat avokysymykset:
\begin{itemize}
\item Saatko sellaista apua, mitä kaipaat? Minkälaista apua kaipaat matematiikassa?
\item Mitkä ovat vahvuutesi matematiikassa? Entä heikkoudet?
\item Kuinka harjoituskurssia voisi parantaa?
\item Miksi päätit osallistua tälle kurssille?
\item Toivoisitko vastaavanlaista kurssia myös myöhempien opintojen aikana?
\end{itemize}
\begin{comment}
Haastattelimme myös harjoituskurssin opettajaa.
\textbf{Tämä pitäisi vielä tehdä..}
\begin{itemize}
\item Miten autat oppilasta löytämään ratkaisun?
\item Millainen on tyypillinen laskupajan asiakas?
\item Onko laskupaja yleensä ollut suosittu?
\item Minkälaisille opiskelijoille laskupaja on suunnattu?
\item Kuinka hyödylliseksi koet laskupajan?
\end{itemize}
\end{comment}

